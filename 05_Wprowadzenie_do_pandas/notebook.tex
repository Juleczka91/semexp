
% Default to the notebook output style

    


% Inherit from the specified cell style.




    
\documentclass[11pt]{article}

    
    
    \usepackage[T1]{fontenc}
    % Nicer default font (+ math font) than Computer Modern for most use cases
    \usepackage{mathpazo}

    % Basic figure setup, for now with no caption control since it's done
    % automatically by Pandoc (which extracts ![](path) syntax from Markdown).
    \usepackage{graphicx}
    % We will generate all images so they have a width \maxwidth. This means
    % that they will get their normal width if they fit onto the page, but
    % are scaled down if they would overflow the margins.
    \makeatletter
    \def\maxwidth{\ifdim\Gin@nat@width>\linewidth\linewidth
    \else\Gin@nat@width\fi}
    \makeatother
    \let\Oldincludegraphics\includegraphics
    % Set max figure width to be 80% of text width, for now hardcoded.
    \renewcommand{\includegraphics}[1]{\Oldincludegraphics[width=.8\maxwidth]{#1}}
    % Ensure that by default, figures have no caption (until we provide a
    % proper Figure object with a Caption API and a way to capture that
    % in the conversion process - todo).
    \usepackage{caption}
    \DeclareCaptionLabelFormat{nolabel}{}
    \captionsetup{labelformat=nolabel}

    \usepackage{adjustbox} % Used to constrain images to a maximum size 
    \usepackage{xcolor} % Allow colors to be defined
    \usepackage{enumerate} % Needed for markdown enumerations to work
    \usepackage{geometry} % Used to adjust the document margins
    \usepackage{amsmath} % Equations
    \usepackage{amssymb} % Equations
    \usepackage{textcomp} % defines textquotesingle
    % Hack from http://tex.stackexchange.com/a/47451/13684:
    \AtBeginDocument{%
        \def\PYZsq{\textquotesingle}% Upright quotes in Pygmentized code
    }
    \usepackage{upquote} % Upright quotes for verbatim code
    \usepackage{eurosym} % defines \euro
    \usepackage[mathletters]{ucs} % Extended unicode (utf-8) support
    \usepackage[utf8x]{inputenc} % Allow utf-8 characters in the tex document
    \usepackage{fancyvrb} % verbatim replacement that allows latex
    \usepackage{grffile} % extends the file name processing of package graphics 
                         % to support a larger range 
    % The hyperref package gives us a pdf with properly built
    % internal navigation ('pdf bookmarks' for the table of contents,
    % internal cross-reference links, web links for URLs, etc.)
    \usepackage{hyperref}
    \usepackage{longtable} % longtable support required by pandoc >1.10
    \usepackage{booktabs}  % table support for pandoc > 1.12.2
    \usepackage[inline]{enumitem} % IRkernel/repr support (it uses the enumerate* environment)
    \usepackage[normalem]{ulem} % ulem is needed to support strikethroughs (\sout)
                                % normalem makes italics be italics, not underlines
    

    
    
    % Colors for the hyperref package
    \definecolor{urlcolor}{rgb}{0,.145,.698}
    \definecolor{linkcolor}{rgb}{.71,0.21,0.01}
    \definecolor{citecolor}{rgb}{.12,.54,.11}

    % ANSI colors
    \definecolor{ansi-black}{HTML}{3E424D}
    \definecolor{ansi-black-intense}{HTML}{282C36}
    \definecolor{ansi-red}{HTML}{E75C58}
    \definecolor{ansi-red-intense}{HTML}{B22B31}
    \definecolor{ansi-green}{HTML}{00A250}
    \definecolor{ansi-green-intense}{HTML}{007427}
    \definecolor{ansi-yellow}{HTML}{DDB62B}
    \definecolor{ansi-yellow-intense}{HTML}{B27D12}
    \definecolor{ansi-blue}{HTML}{208FFB}
    \definecolor{ansi-blue-intense}{HTML}{0065CA}
    \definecolor{ansi-magenta}{HTML}{D160C4}
    \definecolor{ansi-magenta-intense}{HTML}{A03196}
    \definecolor{ansi-cyan}{HTML}{60C6C8}
    \definecolor{ansi-cyan-intense}{HTML}{258F8F}
    \definecolor{ansi-white}{HTML}{C5C1B4}
    \definecolor{ansi-white-intense}{HTML}{A1A6B2}

    % commands and environments needed by pandoc snippets
    % extracted from the output of `pandoc -s`
    \providecommand{\tightlist}{%
      \setlength{\itemsep}{0pt}\setlength{\parskip}{0pt}}
    \DefineVerbatimEnvironment{Highlighting}{Verbatim}{commandchars=\\\{\}}
    % Add ',fontsize=\small' for more characters per line
    \newenvironment{Shaded}{}{}
    \newcommand{\KeywordTok}[1]{\textcolor[rgb]{0.00,0.44,0.13}{\textbf{{#1}}}}
    \newcommand{\DataTypeTok}[1]{\textcolor[rgb]{0.56,0.13,0.00}{{#1}}}
    \newcommand{\DecValTok}[1]{\textcolor[rgb]{0.25,0.63,0.44}{{#1}}}
    \newcommand{\BaseNTok}[1]{\textcolor[rgb]{0.25,0.63,0.44}{{#1}}}
    \newcommand{\FloatTok}[1]{\textcolor[rgb]{0.25,0.63,0.44}{{#1}}}
    \newcommand{\CharTok}[1]{\textcolor[rgb]{0.25,0.44,0.63}{{#1}}}
    \newcommand{\StringTok}[1]{\textcolor[rgb]{0.25,0.44,0.63}{{#1}}}
    \newcommand{\CommentTok}[1]{\textcolor[rgb]{0.38,0.63,0.69}{\textit{{#1}}}}
    \newcommand{\OtherTok}[1]{\textcolor[rgb]{0.00,0.44,0.13}{{#1}}}
    \newcommand{\AlertTok}[1]{\textcolor[rgb]{1.00,0.00,0.00}{\textbf{{#1}}}}
    \newcommand{\FunctionTok}[1]{\textcolor[rgb]{0.02,0.16,0.49}{{#1}}}
    \newcommand{\RegionMarkerTok}[1]{{#1}}
    \newcommand{\ErrorTok}[1]{\textcolor[rgb]{1.00,0.00,0.00}{\textbf{{#1}}}}
    \newcommand{\NormalTok}[1]{{#1}}
    
    % Additional commands for more recent versions of Pandoc
    \newcommand{\ConstantTok}[1]{\textcolor[rgb]{0.53,0.00,0.00}{{#1}}}
    \newcommand{\SpecialCharTok}[1]{\textcolor[rgb]{0.25,0.44,0.63}{{#1}}}
    \newcommand{\VerbatimStringTok}[1]{\textcolor[rgb]{0.25,0.44,0.63}{{#1}}}
    \newcommand{\SpecialStringTok}[1]{\textcolor[rgb]{0.73,0.40,0.53}{{#1}}}
    \newcommand{\ImportTok}[1]{{#1}}
    \newcommand{\DocumentationTok}[1]{\textcolor[rgb]{0.73,0.13,0.13}{\textit{{#1}}}}
    \newcommand{\AnnotationTok}[1]{\textcolor[rgb]{0.38,0.63,0.69}{\textbf{\textit{{#1}}}}}
    \newcommand{\CommentVarTok}[1]{\textcolor[rgb]{0.38,0.63,0.69}{\textbf{\textit{{#1}}}}}
    \newcommand{\VariableTok}[1]{\textcolor[rgb]{0.10,0.09,0.49}{{#1}}}
    \newcommand{\ControlFlowTok}[1]{\textcolor[rgb]{0.00,0.44,0.13}{\textbf{{#1}}}}
    \newcommand{\OperatorTok}[1]{\textcolor[rgb]{0.40,0.40,0.40}{{#1}}}
    \newcommand{\BuiltInTok}[1]{{#1}}
    \newcommand{\ExtensionTok}[1]{{#1}}
    \newcommand{\PreprocessorTok}[1]{\textcolor[rgb]{0.74,0.48,0.00}{{#1}}}
    \newcommand{\AttributeTok}[1]{\textcolor[rgb]{0.49,0.56,0.16}{{#1}}}
    \newcommand{\InformationTok}[1]{\textcolor[rgb]{0.38,0.63,0.69}{\textbf{\textit{{#1}}}}}
    \newcommand{\WarningTok}[1]{\textcolor[rgb]{0.38,0.63,0.69}{\textbf{\textit{{#1}}}}}
    
    
    % Define a nice break command that doesn't care if a line doesn't already
    % exist.
    \def\br{\hspace*{\fill} \\* }
    % Math Jax compatability definitions
    \def\gt{>}
    \def\lt{<}
    % Document parameters
    \title{Operacje na ramkach danych}
    
    
    

    % Pygments definitions
    
\makeatletter
\def\PY@reset{\let\PY@it=\relax \let\PY@bf=\relax%
    \let\PY@ul=\relax \let\PY@tc=\relax%
    \let\PY@bc=\relax \let\PY@ff=\relax}
\def\PY@tok#1{\csname PY@tok@#1\endcsname}
\def\PY@toks#1+{\ifx\relax#1\empty\else%
    \PY@tok{#1}\expandafter\PY@toks\fi}
\def\PY@do#1{\PY@bc{\PY@tc{\PY@ul{%
    \PY@it{\PY@bf{\PY@ff{#1}}}}}}}
\def\PY#1#2{\PY@reset\PY@toks#1+\relax+\PY@do{#2}}

\expandafter\def\csname PY@tok@w\endcsname{\def\PY@tc##1{\textcolor[rgb]{0.73,0.73,0.73}{##1}}}
\expandafter\def\csname PY@tok@c\endcsname{\let\PY@it=\textit\def\PY@tc##1{\textcolor[rgb]{0.25,0.50,0.50}{##1}}}
\expandafter\def\csname PY@tok@cp\endcsname{\def\PY@tc##1{\textcolor[rgb]{0.74,0.48,0.00}{##1}}}
\expandafter\def\csname PY@tok@k\endcsname{\let\PY@bf=\textbf\def\PY@tc##1{\textcolor[rgb]{0.00,0.50,0.00}{##1}}}
\expandafter\def\csname PY@tok@kp\endcsname{\def\PY@tc##1{\textcolor[rgb]{0.00,0.50,0.00}{##1}}}
\expandafter\def\csname PY@tok@kt\endcsname{\def\PY@tc##1{\textcolor[rgb]{0.69,0.00,0.25}{##1}}}
\expandafter\def\csname PY@tok@o\endcsname{\def\PY@tc##1{\textcolor[rgb]{0.40,0.40,0.40}{##1}}}
\expandafter\def\csname PY@tok@ow\endcsname{\let\PY@bf=\textbf\def\PY@tc##1{\textcolor[rgb]{0.67,0.13,1.00}{##1}}}
\expandafter\def\csname PY@tok@nb\endcsname{\def\PY@tc##1{\textcolor[rgb]{0.00,0.50,0.00}{##1}}}
\expandafter\def\csname PY@tok@nf\endcsname{\def\PY@tc##1{\textcolor[rgb]{0.00,0.00,1.00}{##1}}}
\expandafter\def\csname PY@tok@nc\endcsname{\let\PY@bf=\textbf\def\PY@tc##1{\textcolor[rgb]{0.00,0.00,1.00}{##1}}}
\expandafter\def\csname PY@tok@nn\endcsname{\let\PY@bf=\textbf\def\PY@tc##1{\textcolor[rgb]{0.00,0.00,1.00}{##1}}}
\expandafter\def\csname PY@tok@ne\endcsname{\let\PY@bf=\textbf\def\PY@tc##1{\textcolor[rgb]{0.82,0.25,0.23}{##1}}}
\expandafter\def\csname PY@tok@nv\endcsname{\def\PY@tc##1{\textcolor[rgb]{0.10,0.09,0.49}{##1}}}
\expandafter\def\csname PY@tok@no\endcsname{\def\PY@tc##1{\textcolor[rgb]{0.53,0.00,0.00}{##1}}}
\expandafter\def\csname PY@tok@nl\endcsname{\def\PY@tc##1{\textcolor[rgb]{0.63,0.63,0.00}{##1}}}
\expandafter\def\csname PY@tok@ni\endcsname{\let\PY@bf=\textbf\def\PY@tc##1{\textcolor[rgb]{0.60,0.60,0.60}{##1}}}
\expandafter\def\csname PY@tok@na\endcsname{\def\PY@tc##1{\textcolor[rgb]{0.49,0.56,0.16}{##1}}}
\expandafter\def\csname PY@tok@nt\endcsname{\let\PY@bf=\textbf\def\PY@tc##1{\textcolor[rgb]{0.00,0.50,0.00}{##1}}}
\expandafter\def\csname PY@tok@nd\endcsname{\def\PY@tc##1{\textcolor[rgb]{0.67,0.13,1.00}{##1}}}
\expandafter\def\csname PY@tok@s\endcsname{\def\PY@tc##1{\textcolor[rgb]{0.73,0.13,0.13}{##1}}}
\expandafter\def\csname PY@tok@sd\endcsname{\let\PY@it=\textit\def\PY@tc##1{\textcolor[rgb]{0.73,0.13,0.13}{##1}}}
\expandafter\def\csname PY@tok@si\endcsname{\let\PY@bf=\textbf\def\PY@tc##1{\textcolor[rgb]{0.73,0.40,0.53}{##1}}}
\expandafter\def\csname PY@tok@se\endcsname{\let\PY@bf=\textbf\def\PY@tc##1{\textcolor[rgb]{0.73,0.40,0.13}{##1}}}
\expandafter\def\csname PY@tok@sr\endcsname{\def\PY@tc##1{\textcolor[rgb]{0.73,0.40,0.53}{##1}}}
\expandafter\def\csname PY@tok@ss\endcsname{\def\PY@tc##1{\textcolor[rgb]{0.10,0.09,0.49}{##1}}}
\expandafter\def\csname PY@tok@sx\endcsname{\def\PY@tc##1{\textcolor[rgb]{0.00,0.50,0.00}{##1}}}
\expandafter\def\csname PY@tok@m\endcsname{\def\PY@tc##1{\textcolor[rgb]{0.40,0.40,0.40}{##1}}}
\expandafter\def\csname PY@tok@gh\endcsname{\let\PY@bf=\textbf\def\PY@tc##1{\textcolor[rgb]{0.00,0.00,0.50}{##1}}}
\expandafter\def\csname PY@tok@gu\endcsname{\let\PY@bf=\textbf\def\PY@tc##1{\textcolor[rgb]{0.50,0.00,0.50}{##1}}}
\expandafter\def\csname PY@tok@gd\endcsname{\def\PY@tc##1{\textcolor[rgb]{0.63,0.00,0.00}{##1}}}
\expandafter\def\csname PY@tok@gi\endcsname{\def\PY@tc##1{\textcolor[rgb]{0.00,0.63,0.00}{##1}}}
\expandafter\def\csname PY@tok@gr\endcsname{\def\PY@tc##1{\textcolor[rgb]{1.00,0.00,0.00}{##1}}}
\expandafter\def\csname PY@tok@ge\endcsname{\let\PY@it=\textit}
\expandafter\def\csname PY@tok@gs\endcsname{\let\PY@bf=\textbf}
\expandafter\def\csname PY@tok@gp\endcsname{\let\PY@bf=\textbf\def\PY@tc##1{\textcolor[rgb]{0.00,0.00,0.50}{##1}}}
\expandafter\def\csname PY@tok@go\endcsname{\def\PY@tc##1{\textcolor[rgb]{0.53,0.53,0.53}{##1}}}
\expandafter\def\csname PY@tok@gt\endcsname{\def\PY@tc##1{\textcolor[rgb]{0.00,0.27,0.87}{##1}}}
\expandafter\def\csname PY@tok@err\endcsname{\def\PY@bc##1{\setlength{\fboxsep}{0pt}\fcolorbox[rgb]{1.00,0.00,0.00}{1,1,1}{\strut ##1}}}
\expandafter\def\csname PY@tok@kc\endcsname{\let\PY@bf=\textbf\def\PY@tc##1{\textcolor[rgb]{0.00,0.50,0.00}{##1}}}
\expandafter\def\csname PY@tok@kd\endcsname{\let\PY@bf=\textbf\def\PY@tc##1{\textcolor[rgb]{0.00,0.50,0.00}{##1}}}
\expandafter\def\csname PY@tok@kn\endcsname{\let\PY@bf=\textbf\def\PY@tc##1{\textcolor[rgb]{0.00,0.50,0.00}{##1}}}
\expandafter\def\csname PY@tok@kr\endcsname{\let\PY@bf=\textbf\def\PY@tc##1{\textcolor[rgb]{0.00,0.50,0.00}{##1}}}
\expandafter\def\csname PY@tok@bp\endcsname{\def\PY@tc##1{\textcolor[rgb]{0.00,0.50,0.00}{##1}}}
\expandafter\def\csname PY@tok@fm\endcsname{\def\PY@tc##1{\textcolor[rgb]{0.00,0.00,1.00}{##1}}}
\expandafter\def\csname PY@tok@vc\endcsname{\def\PY@tc##1{\textcolor[rgb]{0.10,0.09,0.49}{##1}}}
\expandafter\def\csname PY@tok@vg\endcsname{\def\PY@tc##1{\textcolor[rgb]{0.10,0.09,0.49}{##1}}}
\expandafter\def\csname PY@tok@vi\endcsname{\def\PY@tc##1{\textcolor[rgb]{0.10,0.09,0.49}{##1}}}
\expandafter\def\csname PY@tok@vm\endcsname{\def\PY@tc##1{\textcolor[rgb]{0.10,0.09,0.49}{##1}}}
\expandafter\def\csname PY@tok@sa\endcsname{\def\PY@tc##1{\textcolor[rgb]{0.73,0.13,0.13}{##1}}}
\expandafter\def\csname PY@tok@sb\endcsname{\def\PY@tc##1{\textcolor[rgb]{0.73,0.13,0.13}{##1}}}
\expandafter\def\csname PY@tok@sc\endcsname{\def\PY@tc##1{\textcolor[rgb]{0.73,0.13,0.13}{##1}}}
\expandafter\def\csname PY@tok@dl\endcsname{\def\PY@tc##1{\textcolor[rgb]{0.73,0.13,0.13}{##1}}}
\expandafter\def\csname PY@tok@s2\endcsname{\def\PY@tc##1{\textcolor[rgb]{0.73,0.13,0.13}{##1}}}
\expandafter\def\csname PY@tok@sh\endcsname{\def\PY@tc##1{\textcolor[rgb]{0.73,0.13,0.13}{##1}}}
\expandafter\def\csname PY@tok@s1\endcsname{\def\PY@tc##1{\textcolor[rgb]{0.73,0.13,0.13}{##1}}}
\expandafter\def\csname PY@tok@mb\endcsname{\def\PY@tc##1{\textcolor[rgb]{0.40,0.40,0.40}{##1}}}
\expandafter\def\csname PY@tok@mf\endcsname{\def\PY@tc##1{\textcolor[rgb]{0.40,0.40,0.40}{##1}}}
\expandafter\def\csname PY@tok@mh\endcsname{\def\PY@tc##1{\textcolor[rgb]{0.40,0.40,0.40}{##1}}}
\expandafter\def\csname PY@tok@mi\endcsname{\def\PY@tc##1{\textcolor[rgb]{0.40,0.40,0.40}{##1}}}
\expandafter\def\csname PY@tok@il\endcsname{\def\PY@tc##1{\textcolor[rgb]{0.40,0.40,0.40}{##1}}}
\expandafter\def\csname PY@tok@mo\endcsname{\def\PY@tc##1{\textcolor[rgb]{0.40,0.40,0.40}{##1}}}
\expandafter\def\csname PY@tok@ch\endcsname{\let\PY@it=\textit\def\PY@tc##1{\textcolor[rgb]{0.25,0.50,0.50}{##1}}}
\expandafter\def\csname PY@tok@cm\endcsname{\let\PY@it=\textit\def\PY@tc##1{\textcolor[rgb]{0.25,0.50,0.50}{##1}}}
\expandafter\def\csname PY@tok@cpf\endcsname{\let\PY@it=\textit\def\PY@tc##1{\textcolor[rgb]{0.25,0.50,0.50}{##1}}}
\expandafter\def\csname PY@tok@c1\endcsname{\let\PY@it=\textit\def\PY@tc##1{\textcolor[rgb]{0.25,0.50,0.50}{##1}}}
\expandafter\def\csname PY@tok@cs\endcsname{\let\PY@it=\textit\def\PY@tc##1{\textcolor[rgb]{0.25,0.50,0.50}{##1}}}

\def\PYZbs{\char`\\}
\def\PYZus{\char`\_}
\def\PYZob{\char`\{}
\def\PYZcb{\char`\}}
\def\PYZca{\char`\^}
\def\PYZam{\char`\&}
\def\PYZlt{\char`\<}
\def\PYZgt{\char`\>}
\def\PYZsh{\char`\#}
\def\PYZpc{\char`\%}
\def\PYZdl{\char`\$}
\def\PYZhy{\char`\-}
\def\PYZsq{\char`\'}
\def\PYZdq{\char`\"}
\def\PYZti{\char`\~}
% for compatibility with earlier versions
\def\PYZat{@}
\def\PYZlb{[}
\def\PYZrb{]}
\makeatother


    % Exact colors from NB
    \definecolor{incolor}{rgb}{0.0, 0.0, 0.5}
    \definecolor{outcolor}{rgb}{0.545, 0.0, 0.0}



    
    % Prevent overflowing lines due to hard-to-break entities
    \sloppy 
    % Setup hyperref package
    \hypersetup{
      breaklinks=true,  % so long urls are correctly broken across lines
      colorlinks=true,
      urlcolor=urlcolor,
      linkcolor=linkcolor,
      citecolor=citecolor,
      }
    % Slightly bigger margins than the latex defaults
    
    \geometry{verbose,tmargin=1in,bmargin=1in,lmargin=1in,rmargin=1in}
    
    

    \begin{document}
    
    
    \maketitle
    
    

    
    \hypertarget{operacje-na-ramkach-danych}{%
\section{Operacje na ramkach danych}\label{operacje-na-ramkach-danych}}

    \begin{Verbatim}[commandchars=\\\{\}]
{\color{incolor}In [{\color{incolor}77}]:} \PY{k+kn}{import} \PY{n+nn}{pandas} \PY{k}{as} \PY{n+nn}{pd}
         \PY{k+kn}{from} \PY{n+nn}{pandas} \PY{k}{import} \PY{n}{DataFrame}\PY{p}{,} \PY{n}{Series}
         \PY{k+kn}{from} \PY{n+nn}{numpy} \PY{k}{import} \PY{n}{nan}
\end{Verbatim}


    \hypertarget{usuwanie-brakujux105cych-obserwacji}{%
\subsection{Usuwanie brakujących
obserwacji}\label{usuwanie-brakujux105cych-obserwacji}}

Czasami w naszym zbiorze danych znajdują się brakujące wartości. W takim
wypadku chcielibyśmy je zidentyfikować oraz usunąć. W pakiecie
\texttt{pandas} brakujące wartości oznaczane są obiektem \texttt{nan},
co jest skrótem od ``not a number''. W R odpowiednikiem jest wartość
\texttt{NA}. Przy wczytywaniu danych warto zadbać o to, żeby prawidłowo
wczytać również te obserwacje, które są niekompletne (więcej o tym w
notebooku poświęconym wczytywaniu danych). Na potrzeby ćwiczenia
stworzymy ramkę danych z brakującymi wartościami.

    \begin{Verbatim}[commandchars=\\\{\}]
{\color{incolor}In [{\color{incolor}78}]:} \PY{n}{data} \PY{o}{=} \PY{n}{DataFrame}\PY{p}{(}\PY{p}{\PYZob{}}
             \PY{l+s+s1}{\PYZsq{}}\PY{l+s+s1}{uczestnik}\PY{l+s+s1}{\PYZsq{}}\PY{p}{:} \PY{p}{[}\PY{l+m+mi}{1}\PY{p}{,}\PY{l+m+mi}{2}\PY{p}{,}\PY{l+m+mi}{3}\PY{p}{,}\PY{l+m+mi}{4}\PY{p}{,}\PY{l+m+mi}{5}\PY{p}{,}\PY{l+m+mi}{6}\PY{p}{,}\PY{l+m+mi}{7}\PY{p}{,}\PY{l+m+mi}{8}\PY{p}{,}\PY{l+m+mi}{9}\PY{p}{,}\PY{l+m+mi}{10}\PY{p}{]}\PY{p}{,}
             \PY{l+s+s1}{\PYZsq{}}\PY{l+s+s1}{wiek}\PY{l+s+s1}{\PYZsq{}} \PY{p}{:} \PY{p}{[}\PY{l+m+mi}{26}\PY{p}{,}\PY{l+m+mi}{34}\PY{p}{,}\PY{n}{nan}\PY{p}{,}\PY{l+m+mi}{25}\PY{p}{,}\PY{l+m+mi}{56}\PY{p}{,}\PY{n}{nan}\PY{p}{,}\PY{l+m+mi}{23}\PY{p}{,}\PY{l+m+mi}{21}\PY{p}{,}\PY{l+m+mi}{19}\PY{p}{,}\PY{l+m+mi}{10}\PY{p}{]}\PY{p}{,}
             \PY{l+s+s1}{\PYZsq{}}\PY{l+s+s1}{płeć}\PY{l+s+s1}{\PYZsq{}} \PY{p}{:} \PY{p}{[}\PY{l+s+s1}{\PYZsq{}}\PY{l+s+s1}{K}\PY{l+s+s1}{\PYZsq{}}\PY{p}{,}\PY{l+s+s1}{\PYZsq{}}\PY{l+s+s1}{M}\PY{l+s+s1}{\PYZsq{}}\PY{p}{,}\PY{l+s+s1}{\PYZsq{}}\PY{l+s+s1}{K}\PY{l+s+s1}{\PYZsq{}}\PY{p}{,}\PY{l+s+s1}{\PYZsq{}}\PY{l+s+s1}{M}\PY{l+s+s1}{\PYZsq{}}\PY{p}{,}\PY{n}{nan}\PY{p}{,}\PY{l+s+s1}{\PYZsq{}}\PY{l+s+s1}{K}\PY{l+s+s1}{\PYZsq{}}\PY{p}{,}\PY{l+s+s1}{\PYZsq{}}\PY{l+s+s1}{M}\PY{l+s+s1}{\PYZsq{}}\PY{p}{,}\PY{l+s+s1}{\PYZsq{}}\PY{l+s+s1}{K}\PY{l+s+s1}{\PYZsq{}}\PY{p}{,} \PY{n}{nan}\PY{p}{,} \PY{l+s+s1}{\PYZsq{}}\PY{l+s+s1}{M}\PY{l+s+s1}{\PYZsq{}}\PY{p}{]}\PY{p}{,}
         \PY{p}{\PYZcb{}}\PY{p}{)}
         \PY{n}{data}
\end{Verbatim}


\begin{Verbatim}[commandchars=\\\{\}]
{\color{outcolor}Out[{\color{outcolor}78}]:}    uczestnik  wiek płeć
         0          1  26.0    K
         1          2  34.0    M
         2          3   NaN    K
         3          4  25.0    M
         4          5  56.0  NaN
         5          6   NaN    K
         6          7  23.0    M
         7          8  21.0    K
         8          9  19.0  NaN
         9         10  10.0    M
\end{Verbatim}
            
    Pierwszym naszym zadaniem jest zidentyfikowanie niekompletnych
obserwacji. Możemy posłużyć się metodą \texttt{isna} wywołaną na ramce
danych.

    \begin{Verbatim}[commandchars=\\\{\}]
{\color{incolor}In [{\color{incolor}79}]:} \PY{n}{data}\PY{o}{.}\PY{n}{isna}\PY{p}{(}\PY{p}{)}
\end{Verbatim}


\begin{Verbatim}[commandchars=\\\{\}]
{\color{outcolor}Out[{\color{outcolor}79}]:}    uczestnik   wiek   płeć
         0      False  False  False
         1      False  False  False
         2      False   True  False
         3      False  False  False
         4      False  False   True
         5      False   True  False
         6      False  False  False
         7      False  False  False
         8      False  False   True
         9      False  False  False
\end{Verbatim}
            
    Aby zobaczyć tylko niekompletne obserwacje możemy wykorzystać naszą
wiedzę dotyczącą filtrowania danych. Za pomocą metody \texttt{any} (z
argumentem \texttt{1}, który mówi, że szukamy przynajmniej jednej takiej
wartości wierszami) możemy sprawdzić, w których wierszach znajduje się
przynajmniej jedna brakująca obserwacja:

    \begin{Verbatim}[commandchars=\\\{\}]
{\color{incolor}In [{\color{incolor}80}]:} \PY{n}{data}\PY{o}{.}\PY{n}{isna}\PY{p}{(}\PY{p}{)}\PY{o}{.}\PY{n}{any}\PY{p}{(}\PY{l+m+mi}{1}\PY{p}{)}
\end{Verbatim}


\begin{Verbatim}[commandchars=\\\{\}]
{\color{outcolor}Out[{\color{outcolor}80}]:} 0    False
         1    False
         2     True
         3    False
         4     True
         5     True
         6    False
         7    False
         8     True
         9    False
         dtype: bool
\end{Verbatim}
            
    Teraz możemy wyświetlić tylko niekompletne obserwacje i np. przyjrzeć
się im bliżej:

    \begin{Verbatim}[commandchars=\\\{\}]
{\color{incolor}In [{\color{incolor}81}]:} \PY{n}{data}\PY{p}{[}\PY{n}{data}\PY{o}{.}\PY{n}{isna}\PY{p}{(}\PY{p}{)}\PY{o}{.}\PY{n}{any}\PY{p}{(}\PY{l+m+mi}{1}\PY{p}{)}\PY{p}{]}
\end{Verbatim}


\begin{Verbatim}[commandchars=\\\{\}]
{\color{outcolor}Out[{\color{outcolor}81}]:}    uczestnik  wiek płeć
         2          3   NaN    K
         4          5  56.0  NaN
         5          6   NaN    K
         8          9  19.0  NaN
\end{Verbatim}
            
    Jeżeli chcemy po prostu usunąć wszystkie wiersze, w których mamy
przynajmniej jedną brakujacą wartość, wystarczy wywołać na ramce danych
metodę \texttt{dropna}.

    \begin{Verbatim}[commandchars=\\\{\}]
{\color{incolor}In [{\color{incolor}82}]:} \PY{n}{data}\PY{o}{.}\PY{n}{dropna}\PY{p}{(}\PY{p}{)}
\end{Verbatim}


\begin{Verbatim}[commandchars=\\\{\}]
{\color{outcolor}Out[{\color{outcolor}82}]:}    uczestnik  wiek płeć
         0          1  26.0    K
         1          2  34.0    M
         3          4  25.0    M
         6          7  23.0    M
         7          8  21.0    K
         9         10  10.0    M
\end{Verbatim}
            
    Czasami może się jednak zdarzyć, że za niekompletne obserwacje nie
chcemy uważać wszystkich wierszy, w których brakuje jakiejkolwiek
wartości. Powiedzmy, że interesują nas takie wiersze, w których brakuje
co najwyżej jednej wartości. Możemy przekazać w metodzie \texttt{dropna}
argument \texttt{thresh}, dzięki któremu ustalimy odpowiedni próg. W
poniższym przykładzie \texttt{thresh=2} oznacza, że chcemy uzyskać tylko
te obserwacje, gdzie w przynajmniej 2 kolumnach nie mamy brakujących
wartości.

    \begin{Verbatim}[commandchars=\\\{\}]
{\color{incolor}In [{\color{incolor}83}]:} \PY{n}{data}\PY{o}{.}\PY{n}{dropna}\PY{p}{(}\PY{n}{thresh}\PY{o}{=}\PY{l+m+mi}{2}\PY{p}{)}
\end{Verbatim}


\begin{Verbatim}[commandchars=\\\{\}]
{\color{outcolor}Out[{\color{outcolor}83}]:}    uczestnik  wiek płeć
         0          1  26.0    K
         1          2  34.0    M
         2          3   NaN    K
         3          4  25.0    M
         4          5  56.0  NaN
         5          6   NaN    K
         6          7  23.0    M
         7          8  21.0    K
         8          9  19.0  NaN
         9         10  10.0    M
\end{Verbatim}
            
    Może się również zdarzyć tak, że niektóre kolumny mniej nas interesują
niż inne. Załóżmy, ze w naszym zbiorze mamy także kolumnę, w której są
nieobowiązkowe komentarze badanych:

    \begin{Verbatim}[commandchars=\\\{\}]
{\color{incolor}In [{\color{incolor}84}]:} \PY{n}{data}\PY{p}{[}\PY{l+s+s1}{\PYZsq{}}\PY{l+s+s1}{komentarz }\PY{l+s+s1}{\PYZsq{}}\PY{p}{]} \PY{o}{=} \PY{p}{[}\PY{l+s+s1}{\PYZsq{}}\PY{l+s+s1}{Fajne badanie!}\PY{l+s+s1}{\PYZsq{}}\PY{p}{,} 
                             \PY{l+s+s1}{\PYZsq{}}\PY{l+s+s1}{Brak}\PY{l+s+s1}{\PYZsq{}}\PY{p}{,} 
                             \PY{l+s+s1}{\PYZsq{}}\PY{l+s+s1}{HELLO}\PY{l+s+s1}{\PYZsq{}}\PY{p}{,} 
                             \PY{n}{nan}\PY{p}{,} 
                             \PY{l+s+s1}{\PYZsq{}}\PY{l+s+s1}{Trochę długa była ta ankieta}\PY{l+s+s1}{\PYZsq{}}\PY{p}{,} 
                             \PY{n}{nan}\PY{p}{,}
                             \PY{n}{nan}\PY{p}{,}
                             \PY{l+s+s1}{\PYZsq{}}\PY{l+s+s1}{Nie mam}\PY{l+s+s1}{\PYZsq{}}\PY{p}{,}
                             \PY{n}{nan}\PY{p}{,}
                             \PY{l+s+s1}{\PYZsq{}}\PY{l+s+s1}{pzdr}\PY{l+s+s1}{\PYZsq{}}\PY{p}{]}
         \PY{n}{data}
\end{Verbatim}


\begin{Verbatim}[commandchars=\\\{\}]
{\color{outcolor}Out[{\color{outcolor}84}]:}    uczestnik  wiek płeć                    komentarz 
         0          1  26.0    K                Fajne badanie!
         1          2  34.0    M                          Brak
         2          3   NaN    K                         HELLO
         3          4  25.0    M                           NaN
         4          5  56.0  NaN  Trochę długa była ta ankieta
         5          6   NaN    K                           NaN
         6          7  23.0    M                           NaN
         7          8  21.0    K                       Nie mam
         8          9  19.0  NaN                           NaN
         9         10  10.0    M                          pzdr
\end{Verbatim}
            
    Przy odsiewaniu niekompletnych obserwacji chcielibyśmy zignorować
kolumnę \texttt{comment}. W tym celu możemy po prostu wskazać, które
kolumny mają być brane pod uwagę przy decydowaniu, które obserwacje
uznane mają być za niekompletne. Aby to zrobić, musimy jako argument
\texttt{subset} przekazać listę z nazwami relewantnych kolumn:

    \begin{Verbatim}[commandchars=\\\{\}]
{\color{incolor}In [{\color{incolor}85}]:} \PY{n}{data}\PY{o}{.}\PY{n}{dropna}\PY{p}{(}\PY{n}{subset} \PY{o}{=} \PY{p}{[}\PY{l+s+s1}{\PYZsq{}}\PY{l+s+s1}{płeć}\PY{l+s+s1}{\PYZsq{}}\PY{p}{,} \PY{l+s+s1}{\PYZsq{}}\PY{l+s+s1}{wiek}\PY{l+s+s1}{\PYZsq{}}\PY{p}{]}\PY{p}{)}
\end{Verbatim}


\begin{Verbatim}[commandchars=\\\{\}]
{\color{outcolor}Out[{\color{outcolor}85}]:}    uczestnik  wiek płeć      komentarz 
         0          1  26.0    K  Fajne badanie!
         1          2  34.0    M            Brak
         3          4  25.0    M             NaN
         6          7  23.0    M             NaN
         7          8  21.0    K         Nie mam
         9         10  10.0    M            pzdr
\end{Verbatim}
            
    \hypertarget{zmiana-nazw-kolumn}{%
\subsection{Zmiana nazw kolumn}\label{zmiana-nazw-kolumn}}

    Załóżmy, że nasze dane chcielibyśmy mieć zapisane w dwóch językach - po
polsku i po angielsku. Aby to osiągnąć musimy zmienić nazwy kolumn. W
tym celu moglibyśmy po prostu nadpisać je ustawiając wartość zmiennej
\texttt{data.columns}. Jeśli chcemy to zrobić selektywnie, lepiej
posłużyć się metodą \texttt{rename} obiektów klasy \texttt{DataFrame}. W
argumencie \texttt{columns} możemy przekazać słownik, w którym klucze są
starymi nazwami a wartości nowymi:

    \begin{Verbatim}[commandchars=\\\{\}]
{\color{incolor}In [{\color{incolor}86}]:} \PY{n}{data}\PY{o}{.}\PY{n}{rename}\PY{p}{(}\PY{n}{columns}\PY{o}{=} \PY{p}{\PYZob{}}\PY{l+s+s1}{\PYZsq{}}\PY{l+s+s1}{uczestnik}\PY{l+s+s1}{\PYZsq{}} \PY{p}{:} \PY{l+s+s1}{\PYZsq{}}\PY{l+s+s1}{participant}\PY{l+s+s1}{\PYZsq{}}\PY{p}{,}
                              \PY{l+s+s1}{\PYZsq{}}\PY{l+s+s1}{wiek}\PY{l+s+s1}{\PYZsq{}} \PY{p}{:} \PY{l+s+s1}{\PYZsq{}}\PY{l+s+s1}{age}\PY{l+s+s1}{\PYZsq{}}\PY{p}{,}
                              \PY{l+s+s1}{\PYZsq{}}\PY{l+s+s1}{płeć}\PY{l+s+s1}{\PYZsq{}} \PY{p}{:} \PY{l+s+s1}{\PYZsq{}}\PY{l+s+s1}{sex}\PY{l+s+s1}{\PYZsq{}}\PY{p}{,}
                              \PY{l+s+s1}{\PYZsq{}}\PY{l+s+s1}{komentarz}\PY{l+s+s1}{\PYZsq{}} \PY{p}{:} \PY{l+s+s1}{\PYZsq{}}\PY{l+s+s1}{comment}\PY{l+s+s1}{\PYZsq{}}\PY{p}{\PYZcb{}}\PY{p}{,}
                    \PY{n}{inplace}\PY{o}{=}\PY{k+kc}{True}\PY{p}{)} \PY{c+c1}{\PYZsh{} domyślnie metody te zwracają nową ramkę, inplace pozwala zmodyfikować już istniejącą}
         \PY{n}{data}
\end{Verbatim}


\begin{Verbatim}[commandchars=\\\{\}]
{\color{outcolor}Out[{\color{outcolor}86}]:}    participant   age  sex                    komentarz 
         0            1  26.0    K                Fajne badanie!
         1            2  34.0    M                          Brak
         2            3   NaN    K                         HELLO
         3            4  25.0    M                           NaN
         4            5  56.0  NaN  Trochę długa była ta ankieta
         5            6   NaN    K                           NaN
         6            7  23.0    M                           NaN
         7            8  21.0    K                       Nie mam
         8            9  19.0  NaN                           NaN
         9           10  10.0    M                          pzdr
\end{Verbatim}
            
    \hypertarget{przekodowywanie-wartoux15bci-w-ramce-danych}{%
\subsection{Przekodowywanie wartości w ramce
danych}\label{przekodowywanie-wartoux15bci-w-ramce-danych}}

    Metoda \texttt{map} obiektów klasy \texttt{Series} pozwala na bardzo
intuicyjne ``przekodowywanie'' wartości w interesujących nas kolumnach.
Możemy na przykłąd zmienić wszystkie wartośći \texttt{K} na
\texttt{Female} oraz \texttt{M} na \texttt{Male}. Sposób działania tej
funkcji jest bardzo podobny, co w przypadku \texttt{rename} - możemy
przekazać odpowiedni słownik.

    \begin{Verbatim}[commandchars=\\\{\}]
{\color{incolor}In [{\color{incolor}88}]:} \PY{n}{data}\PY{p}{[}\PY{l+s+s1}{\PYZsq{}}\PY{l+s+s1}{sex}\PY{l+s+s1}{\PYZsq{}}\PY{p}{]}\PY{o}{.}\PY{n}{map}\PY{p}{(}\PY{p}{\PYZob{}}\PY{l+s+s1}{\PYZsq{}}\PY{l+s+s1}{K}\PY{l+s+s1}{\PYZsq{}} \PY{p}{:} \PY{l+s+s1}{\PYZsq{}}\PY{l+s+s1}{Female}\PY{l+s+s1}{\PYZsq{}}\PY{p}{,} \PY{l+s+s1}{\PYZsq{}}\PY{l+s+s1}{M}\PY{l+s+s1}{\PYZsq{}} \PY{p}{:} \PY{l+s+s1}{\PYZsq{}}\PY{l+s+s1}{Male}\PY{l+s+s1}{\PYZsq{}}\PY{p}{\PYZcb{}}\PY{p}{)}
\end{Verbatim}


\begin{Verbatim}[commandchars=\\\{\}]
{\color{outcolor}Out[{\color{outcolor}88}]:} 0    Female
         1      Male
         2    Female
         3      Male
         4       NaN
         5    Female
         6      Male
         7    Female
         8       NaN
         9      Male
         Name: sex, dtype: object
\end{Verbatim}
            
    Jeżeli interesują nas bardziej zaawansowane operacje, możemy zamiast
słownika przekazać tej metodzie funkcje. Wtedy funkcja ta będzie
aplikowana do każdego elementu \texttt{Series}. W poniższym przykładzie
chcielibyśmy wszystkie wartości z kolumny \texttt{sex} zmienić na małe
litery i zastąpić je wartością ``brak danych'' tam, gdzie występuje
wartość \texttt{nan}:

    \begin{Verbatim}[commandchars=\\\{\}]
{\color{incolor}In [{\color{incolor}87}]:} \PY{n}{data}\PY{p}{[}\PY{l+s+s1}{\PYZsq{}}\PY{l+s+s1}{sex}\PY{l+s+s1}{\PYZsq{}}\PY{p}{]}\PY{o}{.}\PY{n}{map}\PY{p}{(}\PY{k}{lambda} \PY{n}{x}\PY{p}{:} \PY{n}{x}\PY{o}{.}\PY{n}{lower}\PY{p}{(}\PY{p}{)} \PY{k}{if} \PY{n+nb}{type}\PY{p}{(}\PY{n}{x}\PY{p}{)} \PY{o}{==} \PY{n+nb}{str} \PY{k}{else} \PY{l+s+s1}{\PYZsq{}}\PY{l+s+s1}{brak danych}\PY{l+s+s1}{\PYZsq{}}\PY{p}{)}
\end{Verbatim}


\begin{Verbatim}[commandchars=\\\{\}]
{\color{outcolor}Out[{\color{outcolor}87}]:} 0              k
         1              m
         2              k
         3              m
         4    brak danych
         5              k
         6              m
         7              k
         8    brak danych
         9              m
         Name: sex, dtype: object
\end{Verbatim}
            
    \hypertarget{ux142ux105czenie-ramek-danych}{%
\subsection{Łączenie ramek danych}\label{ux142ux105czenie-ramek-danych}}

    \texttt{pandas} oferuje bardzo rozbudowane możliwości złączania ramek
danych (zewnętrzne, wewnętrzne, lewostronne i prawostronne otwarte).
Zainteresowanych odsyłam do dokumentacji (wykracza to BARDZO poza
tematykę kursu):

https://pandas.pydata.org/pandas-docs/stable/reference/api/pandas.DataFrame.join.html

My zajmiemy się najprostszym łączeniem ramek danych za pomocą funkcji
\texttt{concat}. Załóżmy, że dla każdego badanego mamy osobny plik z
danymi (a więc i osobną~ramkę~danych). Jest to sytuacja bardzo częsta
(np. PsychoPy w taki sposób zapisuje dane).

    \begin{Verbatim}[commandchars=\\\{\}]
{\color{incolor}In [{\color{incolor}101}]:} \PY{n}{P1} \PY{o}{=} \PY{n}{DataFrame}\PY{p}{(}\PY{p}{\PYZob{}}
              \PY{l+s+s1}{\PYZsq{}}\PY{l+s+s1}{participant}\PY{l+s+s1}{\PYZsq{}} \PY{p}{:} \PY{p}{[}\PY{l+m+mi}{1}\PY{p}{]} \PY{o}{*} \PY{l+m+mi}{3}\PY{p}{,}
              \PY{l+s+s1}{\PYZsq{}}\PY{l+s+s1}{trialN}\PY{l+s+s1}{\PYZsq{}} \PY{p}{:} \PY{n+nb}{range}\PY{p}{(}\PY{l+m+mi}{1}\PY{p}{,}\PY{l+m+mi}{4}\PY{p}{)}\PY{p}{,}
              \PY{l+s+s1}{\PYZsq{}}\PY{l+s+s1}{rt}\PY{l+s+s1}{\PYZsq{}} \PY{p}{:} \PY{p}{[}\PY{l+m+mi}{345}\PY{p}{,} \PY{l+m+mi}{321}\PY{p}{,} \PY{l+m+mi}{287}\PY{p}{]}
          \PY{p}{\PYZcb{}}\PY{p}{)}
          
          \PY{n}{P2} \PY{o}{=} \PY{n}{DataFrame}\PY{p}{(}\PY{p}{\PYZob{}}
              \PY{l+s+s1}{\PYZsq{}}\PY{l+s+s1}{participant}\PY{l+s+s1}{\PYZsq{}} \PY{p}{:} \PY{p}{[}\PY{l+m+mi}{2}\PY{p}{]} \PY{o}{*} \PY{l+m+mi}{3}\PY{p}{,}
              \PY{l+s+s1}{\PYZsq{}}\PY{l+s+s1}{trialN}\PY{l+s+s1}{\PYZsq{}} \PY{p}{:} \PY{n+nb}{range}\PY{p}{(}\PY{l+m+mi}{1}\PY{p}{,}\PY{l+m+mi}{4}\PY{p}{)}\PY{p}{,}
              \PY{l+s+s1}{\PYZsq{}}\PY{l+s+s1}{rt}\PY{l+s+s1}{\PYZsq{}} \PY{p}{:} \PY{p}{[}\PY{l+m+mi}{213}\PY{p}{,} \PY{l+m+mi}{432}\PY{p}{,} \PY{l+m+mi}{553}\PY{p}{]}
          \PY{p}{\PYZcb{}}\PY{p}{)}
\end{Verbatim}


    \begin{Verbatim}[commandchars=\\\{\}]
{\color{incolor}In [{\color{incolor}102}]:} \PY{n}{P1}
\end{Verbatim}


\begin{Verbatim}[commandchars=\\\{\}]
{\color{outcolor}Out[{\color{outcolor}102}]:}    participant  trialN   rt
          0            1       1  345
          1            1       2  321
          2            1       3  287
\end{Verbatim}
            
    \begin{Verbatim}[commandchars=\\\{\}]
{\color{incolor}In [{\color{incolor}103}]:} \PY{n}{P2}
\end{Verbatim}


\begin{Verbatim}[commandchars=\\\{\}]
{\color{outcolor}Out[{\color{outcolor}103}]:}    participant  trialN   rt
          0            2       1  213
          1            2       2  432
          2            2       3  553
\end{Verbatim}
            
    Najprostszym sposobem połączenia tych ramek danych jest użycie funkcji
\texttt{concat}. Przyjmuje ona listę, której elementami są~ramki danych.

    \begin{Verbatim}[commandchars=\\\{\}]
{\color{incolor}In [{\color{incolor}105}]:} \PY{n}{data} \PY{o}{=} \PY{n}{pd}\PY{o}{.}\PY{n}{concat}\PY{p}{(}\PY{p}{[}\PY{n}{P1}\PY{p}{,} \PY{n}{P2}\PY{p}{]}\PY{p}{,} 
                    \PY{n}{axis}\PY{o}{=}\PY{l+m+mi}{0}\PY{p}{)} \PY{c+c1}{\PYZsh{} oś w której chcemy połączyć ramki, `1` połączy je w szerz}
          \PY{n}{data}
\end{Verbatim}


\begin{Verbatim}[commandchars=\\\{\}]
{\color{outcolor}Out[{\color{outcolor}105}]:}    participant  trialN   rt
          0            1       1  345
          1            1       2  321
          2            1       3  287
          0            2       1  213
          1            2       2  432
          2            2       3  553
\end{Verbatim}
            
    Po wykonaniu takiej operacji dobrze jest zresetować index (proszę
zwrócić uwagę, że teraz indeksy nie są unikatowe).

    \begin{Verbatim}[commandchars=\\\{\}]
{\color{incolor}In [{\color{incolor}108}]:} \PY{n}{data}\PY{o}{.}\PY{n}{reset\PYZus{}index}\PY{p}{(}\PY{n}{drop}\PY{o}{=}\PY{k+kc}{True}\PY{p}{)} \PY{c+c1}{\PYZsh{} drop=True sprawia, że stary indeks nie staje się dodatkową kolumną}
\end{Verbatim}


\begin{Verbatim}[commandchars=\\\{\}]
{\color{outcolor}Out[{\color{outcolor}108}]:}    participant  trialN   rt
          0            1       1  345
          1            1       2  321
          2            1       3  287
          3            2       1  213
          4            2       2  432
          5            2       3  553
\end{Verbatim}
            

    % Add a bibliography block to the postdoc
    
    
    
    \end{document}
